\documentclass{letter}
\usepackage{amssymb,amsmath}
\usepackage{graphicx}

\oddsidemargin=.2in
\evensidemargin=.2in
\textwidth=5.9in
\topmargin=-.5in
\textheight=9in

\newcommand {\qed}{\mbox{$\Box$}}
\renewcommand {\iff}{\Longleftrightarrow}
\newcommand {\R}{\mathbb{R}}
\newcommand {\N}{\mathbb{N}}
\newcommand {\Q}{\mathbb{Q}}
\newcommand {\Z}{\mathbb{Z}}

\newcommand {\sub}{\mbox{SB}}


\begin{document}
\begin{letter}{}

\date{}

\opening{To whom it may concern,}
I am writing to express my interest in the Israeli Council for Higher Education
(CHE) competitive grant for foreign students willing to cooperate for one year
with an Israeli university. My name is Ondřej Pešek. I am a Ph.D. student at
the Czech Technical University in Prague, currently at the beginning of
the third year of my doctoral studies.

The topic of my Ph.D. is \textit{Possibilities of convolutional neural networks
use for remote sensing image classification} and I conduct the research at
the university under the supervision of Prof. Ing. Ales Cepek, CSc., and
Ing. Martin Landa, Ph.D. The goal of the thesis is to perform systematic
research on the possibilities of use of chosen convolutional neural network
architectures on various selected use cases from the field of remote sensing.
So far, two use cases were selected - the first one being classification of
vegetated areas and multi-class classification of urban scenes with a focus on
the differentiation between the urban and rural vegetation, the other one being
detection and classification of horizontal traffic signs. But the thesis should
ideally contain at least three use cases.

To maximize the performance research objectivity, the focus is paid on
the diversity of topics. Here the role for the Israeli cooperation comes. While
already dealing with European vegetation and urban areas, Israel with its
deserts could be an enriching area of interest. At the Ben Gurion University,
there is a remote sensing lab with staff occasionally cooperating with experts
in machine and deep learning and in their history, research using convolutional
neural networks could be found. As their interest lies in topics like
desertification and cloud detection over deserts utilizing also Israeli
satellite system Ven$\mu$s, I would consider a stay at this lab and cooperation
as an ideal third case for my Ph.D. thesis.

My experience with deep learning and convolutional neural networks started with
my master thesis and later I continued in the field both in my Ph.D. and
professional career in parallel. However, during my studies, I attended only
few real courses on deep learning -- therefore, such work in harness could be
also beneficient for me to lead me through some possible blank spaces that
could appear. The same applies to the area of deserts as I do not have any
experience with data from such areas. On the other hand, my experience and
strategies could also breathe new life into tasks already dealt with in
traditional ways and could result in a release of one or more papers.
Therefore, I find this occasion very enriching for both sides.

\closing{Sincerely,\\
Ondřej Pešek
}

Ondřej Pešek\\
Czech Technical Uniersity in Prague\\
Faculty of Civil Engineering\\
Department of Geomatics\\
ondrej.pesek@fsv.cvut.cz
\end{letter}

\end{document}


